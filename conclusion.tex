\section{Conclusion}

Aided by the increase in availability and reduction in cost of both computing
power and sensing equipments, autonomous driving technologies have seen rapid
progress and maturation in the past couple of decades. This report has provided
a glimpse of the various components that make up an autonomous vehicle software
system, and capture some of the currently available state of the art techniques.
Although, the amount of research and literature in autonomous vehicles has increased
significantly in the last decade, there are still difficult challenges
that have to be solved to not only increase the autonomous driving capabilities
of the vehicles, but also to ensure the safety, reliability, and social and
legal acceptability aspects of autonomous driving.

Autonomous vehicles are complex systems. It is therefore more pragmatic for
researchers to compartmentalize the AV software structure and focus on
advancement of individual subsystems as part of the whole, realizing new
capabilities through improvements to these separate subsystems. A critical but
sometimes overlooked challenge in autonomous system research is the seamless
integration of all these components, ensuring that the interaction between
different software components are meaningful and valid. Due to overall system
complexity, it can also be difficult to guarantee that the sum of local process
intentions results in the desired final output of the system. Balancing
computational resource allotments amongst the various individual processes in
the system is also a key challenge.

Recognizing the fast pace of research advancement in AVs, we eagerly anticipate
the near future developments which will overcome the cited challenges and bring
AVs to greater prevalence in urban transportation systems.